% Mozilla Addon Builder
% Package Building System Documentation
% 
% This document is written in LaTeX
% For quick doc please follow to http://web.mit.edu/olh/Latex/ess:Latex.html

% If you want to edit the text please scroll down to the CONTENT

\documentclass[10pt]{article}
\usepackage{mozillaaddonbuilderstyle}
\def\version{alpha 0.1}

% mozillaaddonbuilderstyle package loads following packages:
% fullpage, url, tikz, tikz-er2, listings, graphicx, color
% 
% Graphz are using the TikZ package: http://texamples.net/tikz
% 
% TikZ-ER2 (Creative Commons Attribution 2.5 Generic License) 
% needs to be installed https://www.assembla.com/wiki/show/tikz-er2 

               
% CONTENT -----------------------------------------------------------------------------                         
                         
\title{Mozilla Addon Builder\\ Package Building System}
\author{Piotr Zalewa}
\date{\today, \version}

\begin{document}
\maketitle

{\scriptsize
	\noindent Download this document from \\
\url{http://github.com/zalun/FlightDeck/raw/master/Docs/Package%20Building%20System.pdf}\\
}{\scriptsize
	\noindent Some relevant graph slides are available \\
	\url{http://github.com/zalun/FlightDeck/raw/master/Docs/Addon%20Builder%20-%20Build%20System.pdf}
}

\section{Assumptions for the current iteration}

	\begin{enumerate}
		\item{Name of the Package is not unique anymore.\\
			Packages are identified by it's {\em unique ID}. There may and probably often will be many
			Packages with the same name.\\
			{\tt /library/123456/}}
		\item{Version is a tag.\\
			Version is important. It is used to tag major {\em Revisions}. If a package is called without any
			Version specified (as above), the latest versioned Revision will be used.\\
			{\tt /library/123456/version/0.1/}}
		\item{Revision Number is used to precisely identify a Revision.\\
			It is completely parallel to the Package Version\\
			{\tt /library/123456/revision/654/}}
		\item{No collaborative editing.\\
			Althought there will be no connection between Packages owned by different Users, design the 
			system to not complicate future implementation of such functionality.}
		\item{Package remembers which SDK version was used to build it.\\
			This is very complicated also on the front-end side. It will be created during the next
			iteration. However, for the future implementations we will save the Jetpack SDK version number 
			which was installed at the moment the Package was created.}
	\end{enumerate}

\section{Data structure}

	\begin{figure}[H]
	\scalebox{1}{\begin{tikzpicture}[node distance=1cm, every edge/.style={link}]
		\node [entity] (User) {User};
		\node [entity] [right=of User] (Package) {Package} edge [->] (User)
			node [attribute] [above=of Package] {ID} edge (Package)
			node [attribute] [above left=of Package] {Version} edge (Package);
		\node [entity] [right=1.5cm of Package] (PackageRevision) {PackageRevision} edge [->] (Package)
			node [attribute] [below=of PackageRevision] {Version} edge [dashed] (PackageRevision)
			node [attribute] (meta) [below left=6mm of PackageRevision] {Meta} edge (PackageRevision)
				node [attribute] [below=of meta] {Name} edge (meta)
				node [attribute] [below left=of meta] {Description} edge (meta);
		\node [relationship] [above=1.5cm of PackageRevision] (depends) {depends on};
		\draw
			(PackageRevision.60) edge [<-] (depends.320) 
			(PackageRevision.130) edge node [rectangle,fill=white,rotate=277] {Library} (depends.220);
		\node [entity] [right=of PackageRevision] (Module) {Module} edge [total] (PackageRevision)
			node [attribute] [right=of Module] {Code} edge (Module)
			node [attribute] [above right=of Module] {Filename} edge (Module);
		\node [entity] [below=of Module] (Attachment) {Attachment} 	edge [total] (PackageRevision)
			node [attribute] [right=of Attachment] {Filename} edge (Attachment)
			node [attribute] [below right=of Attachment] {Upload Path} edge (Attachment);		
	\end{tikzpicture}}
	\caption{	Database design.} 
	\end{figure}
	\begin{itemize}
		\item{Unique filenames for Modules or Attachments are not given by design. Additional code needs to 
			be written to prevent duplicating filenames.}
		\item{{\tt Package:version} is set automatically with setting the {\tt PackageRevision:version}}
	\end{itemize}
	

\section{Export \xpi}

	Only Addons may be used to create an \xpi\footnote{An \xpi\ installer module is a ZIP file 
	that contains a Package --- Manifest with all code needed to run the Addon}. Be aware that it is 
	possible and common to export \xpi\ from partially unsaved data. 
	This happens when User will use the {\em Try in browser} functionality. In this case \xpi\ can not be send to 
	\amo\footnote{\url{http://addons.mozilla.org/}}.
	
	\subsection{Create directory structure}
		
		Directory structure should be as close as standard Jetpack SDK as possible. Jetpack SDK should 
		be copied to a temporary directory as more than one Addon compilation could take action at the 
		same time. Desired revisions of Libraries and Addons will be exported into {\tt packages} directory.
		
		\begin{lstlisting}[caption=Parts of the tree of a copied Jetpack SDK directory.]
 /tmp/jetpack-sdk-{hash}/
 |-- bin/
 |   |-- activate
 |   |-- cfx
 |   `-- [...]
 |-- packages/
 |   |-- jetpack-core/
 |   `-- [...]
 |-- python-lib/
 |-- static-files/
 `-- [...]
		\end{lstlisting}
	
		
	
	\subsection{Export Packages with Modules}
		
		\begin{enumerate}
			\item{Create Package and its Modules directories\\
				{\tt /tmp/jetpack-sdk-\{hash\}/packages/\{Package:ID\}/}\\
				{\tt /tmp/jetpack-sdk-\{hash\}/packages/\{Package:ID\}/lib/}
			}
			\item{Use collected data to create the Manifest.\\
				{\tt /tmp/jetpack-sdk-\{hash\}/packages/\{Package:name\}/package.json}
			}
			\item{Create Module files\\
				Iterate over the assigned Modules and create a "{\tt .js}" file with its content inside 
				Package's {\tt lib/} directory.
			}
			\item{Export dependencies\\
				Iterate over Libraries on which a Package depends and repeat this section ({\em Export 
				the Package with Modules}) for every Library.
			}
		\end{enumerate}
	
	\subsection{Build \xpi}
		\begin{enumerate}
			\item{Set virtual environment to the temporary Jetpack SDK}
			\item{Change directory to {\tt /tmp/jetpack-sdk-\{hash\}/packages/\{Package:name\}/}}
			\item{Call {\tt cfx xpi}.\\
				The {\tt \{Package:name\}.xpi} file will be created in current directory.} 
			\item{Send location to the front-end to be used in further actions\\
				In example calling Test in Browser.}
		\end{enumerate}
		
	\subsection{Test in browser}
		
		To test the Addon in the Browser it is not necessary to save current code in the PackageRevision.
		An \xpi\ will be build from current data displayed inside the Addon Builder\footnote{As UI will 
		probably be designed in the way that the Libraries will be open in different Browser Tab than the 
		Addon, only the latter	data will be taken directly from editor}. Addon will be installed temporarily
		--- it will be automatically removed after certain conditions will happen\footnote{It is not yet 
		decided what these conditions are. Should it happen after User will {\em "leave the editing 
		environment"} or restart the browser}. 
		The process takes following steps:
		\begin{enumerate}
			\item{Build an \xpi\ from code in editor}
			\item{Send a {\em load Addon request} to the {\em FlightDeck Addon}\footnote{FlightDeck Addon is a 
				Jetpack extension 	allowing to temporary installation of the \xpi. It needs to be called 
				with an URL of the \xpi. It is currently in {\em alpha} stage, there is a proposition 
				to replace it with a more generic Addon}}
			\item{\xpi\ is downloaded and installed by the Addon}
			\item{Addon sends a remove request after the \xpi\ is downloaded}
			\item{Whole Jetpack SDK used to create \xpi\ is removed}
		\end{enumerate}
	
	\subsection{Upload to AMO}
	
		{\em Sending the \xpi's to \amo\ is postponed and will be coded after current iteration.}
	
		\noindent \xpi\ needs to be created from a database object. Then {\tt mechanize} lib will
		be used to login to \amo\ and upload the file faking it was done directly from the browser.
							
\section{Integration with the Browser}

	\note{FlightDeck Addon is provided and developed by Mozilla}
	Integration with the Browser is provided by FlightDeck Addon.

	\subsection{Current status}
	
		Currently there is a {\em proof of concept} version of that Addon. It provides the support for instant 
		addons. These are installed temporarily until the browser will be restarted.
		
	\subsection{Try in browser}
	
		Provide a temporary instalaltion of the Addon.
		
		\subsection*{Install Addon}
		
			Addon will download and install an \xpi\ created by the Package Builder. Afterwards it will send
			back a command to remove the SDK. Installation should use the built-in support for instant addons.
		
		\subsection*{Remove Addon}
		
			This should happen on request or automatically after a specific action will happen. It is still
			not decided if leaving the editing environment should be considered as a {\em remove addon} 
			action.
			
			\begin{description}
				\item[On request] means that the User has the information about temporarily installed Addon/s 
					and is able to send request to uninstall it/them.
				\item[On browser quit] --- all of the Addons Iser is currently testing are gonna be removed 
					after Browser will quit or restarted. This should happen after User will quit the Browser,
					terminate it or restart after crash.
				\item[On leaving the editing environment] --- this happens whith the onunload event fired 
					on the Addon tab (navigating away or closing the tab)
				\item[On uninstall/reinstall the FlightDeck Addon] all of the temporarily installed 
					packages should be unloaded. At the moment it requires the browser to be reloaded, but
					it will eventually change if FlightDeck Addon will become a Jetpack addon itself.
			\end{description}
	
\section{Maintaining and building Manifest}

	Manifest is the Addon's metadata. It is exported to {\tt package.json} in the main Addon directory. It is 
	described on \url{https://jetpack.mozillalabs.com/sdk/0.1/docs/\#guide/package-spec}.
	
	\subsection{Manifest's attributes representation in the system}
	
	\begin{description}
		\item[fullName] from {\tt Package:full\_name}
		\item[name] from {\tt Package:name}
		\item[description] from {\tt Package:description} and {\tt PackageRevision:description}
		\item[author] from {\tt Package:author}
		\item[id] from {\tt Package:ID}
		\item[version] depends on the way the \xpi\ was build. It will always contain according 
			{\tt PackageRevision:version\_name} plus {\tt PackageRevision:revision\_number} 
			(if build PackageRevision was not the one containing the version\_name) or  
			{\tt test in browser} (if \xpi\ is build for testing only).
		\item[dependencies] is a list created on the basis of {\tt PackageRevision:dependencies}
		\item[license] from {\tt Package:license}
		\item[url] from {\tt Package:url}
		\item[contributors] {PackageRevision:contributors} --- needs to be a comma separated list as 
			current version will not support collaborative editing 
	\end{description}
	
	
	
\pagebreak
\section*{To be continued\ldots}
\end{document}

