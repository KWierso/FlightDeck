\documentclass[10pt]{article}

% PREAMBULE -------------------------------------------------------------------------
% 
% If you want to edit the text please scroll down to the CONTENT
% 
% This document is written in LaTeX
% For quick doc please follow to http://web.mit.edu/olh/Latex/ess:Latex.html
% All used symbols may be found here: http://www.artofproblemsolving.com/Wiki/index.php/LaTeX:Symbols
% Graphz are using the TikZ package: http://texamples.net/tikz

\usepackage{fullpage, url, tikz, tikz-er2, listings, sidecap, boxedminipage, graphicx, floatrow}
% TikZ is included in the TeXLive distribution
% TikZ-ER2 (Creative Commons Attribution 2.5 Generic License) 
% needs to be installed https://www.assembla.com/wiki/show/tikz-er2 

\DeclareGraphicsExtensions{.pdf,.png,.jpg}

\def\Ua{{\tt Ua}}
\def\Ub{{\tt Ub}}
\def\La{{\tt La}}
\def\Ma{{\tt Ma}}
\def\Mb{{\tt Mb}}

\def\xpi{{\tt XPI}}
\def\amo{{\tt AMO}}

\def\headsto{${\Longrightarrow}$ }
\def\hto{\headsto}
\def\eq{${\supset}$ }

\def\version{alpha 0.1}

\usetikzlibrary{positioning}
\usetikzlibrary{shadows}

\tikzstyle{every entity} = [top color=white, bottom color=blue!30, 
                            draw=blue!50!black!100, drop shadow, rounded corners]
\tikzstyle{every weak entity} = [drop shadow={shadow xshift=.7ex, 
                                 shadow yshift=-.7ex}, rounded corners]
\tikzstyle{every attribute} = [top color=white, bottom color=yellow!20, 
                               draw=yellow, node distance=1cm, drop shadow, rounded corners]
\tikzstyle{every relationship} = [top color=white, bottom color=red!20, minimum size=2.3cm,
                                  draw=red!50!black!100, drop shadow, rounded corners]
\tikzstyle{every isa} = [top color=white, bottom color=green!20, 
                         draw=green!50!black!100, drop shadow, rounded corners]
                         
% CONTENT -----------------------------------------------------------------------------                         
                         
\title{Mozilla Addon Builder\\ Package Building System}
\author{Piotr Zalewa}
\date{\today, \version}

\begin{document}
\maketitle

{\scriptsize
	\noindent Download this document from \\
\url{http://github.com/zalun/FlightDeck/raw/master/Docs/Package%20Building\%20System.pdf}\\
}{\scriptsize
	\noindent If in doubts, please take a look at the accompanied slides at \\
	\url{http://github.com/zalun/FlightDeck/raw/master/Docs/Addon\%20Builder\%20-\%20Build\%20System.pdf}
}

\section{Assumptions for the current iteration}

	\begin{enumerate}
		\item{Name of the Package is not unique anymore.\\
			Packages are identified by it's {\em unique ID}. There may and probably often will be many
			Packages with the same name\footnote{Check if it will not make any problem with Addons and 
			uploading to \amo}.\\
			{\tt /library/123456/}}
		\item{Version is a tag.\\
			Version is important. It is used to tag major {\em Revisions}. If a package is called without any
			Version specified (as above), the latest versioned Revision will be used.\\
			{\tt /library/123456/version/0.1/}}
		\item{Revision Number is used to precisely identify a Revision.\\
			It is completely parallel to the Package Version\\
			{\tt /library/123456/revision/654/}}
		\item{No collaborative editing.\\
			Althought there will be no connection between Packages owned by different Users, design the 
			system to not complicate future implementation of such functionality.}
		\item{Package remembers wich SDK version was used to build it.\\
			This is very complicated also on the front-end side. It will be created during the next
			iteration.}
	\end{enumerate}

\section{Logical structure}

	\scalebox{1}{\begin{tikzpicture}[node distance=1cm, every edge/.style={link}]
		\node [entity] (User) {User};
		\node [entity] [right=of User] (Package) {Package} edge [->] (User)
			node [attribute] [above=of Package] {ID} edge (Package);
		\node [entity] [right=1.5cm of Package] (PackageRevision) {PackageRevision} edge [->] (Package)
			node [attribute] [below=of PackageRevision] {Version} edge [dashed] (PackageRevision)
			node [attribute] (meta) [below left=6mm of PackageRevision] {Meta} edge (PackageRevision)
				node [attribute] [below=of meta] {Name} edge (meta)
				node [attribute] [below left=of meta] {Description} edge (meta);
		\node [relationship] [above=1.5cm of PackageRevision] (depends) {depends on};
		\draw
			(PackageRevision.60) edge [<-] (depends.320) 
			(PackageRevision.130) edge node [rectangle,fill=white,rotate=277] {Library} (depends.220);
		\node [entity] [right=of PackageRevision] (Module) {Module} edge [total] (PackageRevision)
			node [attribute] [right=of Module] {Code} edge (Module);
		\node [entity] [below=of Module] (Attachment) {Attachment} 	edge [total] (PackageRevision)
			node [attribute] [right=of Attachment] {Filename} edge (Attachment);		
	\end{tikzpicture}}


\section{Exporting \xpi}

	Be aware that it is possible and common to export \xpi\footnote{An \xpi\ (pronounced "zippy" and derived 
	from XPInstall) installer module is a ZIP file that contains an install script or a manifest at the root 
	of the file, and a number of data files.} from partially unsaved data. This happens when User will use 
	the "Try in browser" 	functionality. In this case \xpi\ may not be send to 
	\amo\footnote{\url{http://addons.mozilla.org/}}.
	
	\subsection{Creating directory structure}
		
		Directory structure should be as close as standard Jetpack SDK as possible. Jetpack SDK should 
		be copied to a temporary directory as more than one Addon compilation could take action at the 
		same time. Desired revisions of Libraries and Addons will be exported into {\tt packages} directory.

			\begin{figure}[htp,capbesideposition={right,top},capbesideframe=yes]
				\begin{boxedminipage}{6cm}{\footnotesize\begin{lstlisting}		
 /tmp/jetpack-sdk-\{hash\}/
 |-- bin/
 |   |-- activate
 |   |-- cfx
 |   `-- [...]
 |-- packages/
 |   |-- jetpack-core/
 |   `-- [...]
 |-- python-lib/
 |-- static-files/
 `-- [...]
				\end{lstlisting}}\end{boxedminipage}
				\caption{Parts of the tree of a copied Jetpack SDK directory}
			\end{figure}
	
		
	
	\subsection{Export Packages with Modules}
		
		\begin{enumerate}
			\item{Create Package and its Modules directories\\
				{\tt /tmp/jetpack-sdk-\{hash\}/packages/\{Package:name\}/}\\
				{\tt /tmp/jetpack-sdk-\{hash\}/packages/\{Package:name\}/lib/}
			}
			\item{Use collected data to create the Manifest.\\
				{\tt /tmp/jetpack-sdk-\{hash\}/packages/\{Package:name\}/package.json}
			}
			\item{Create Module files\\
				Iterate over the assigned Modules and create a "{\tt .js}" file with its content inside 
				Package's {\tt lib/} directory.
			}
			\item{Export dependencies\\
				Iterate over Libraries on which a Package depends and repeat this section ({\em Export 
				the Package with Modules}) for every Library.
			}
		\end{enumerate}
	
	\subsection{Building \xpi}
		\begin{enumerate}
			\item{Set virtual environment to the temporary Jetpack SDK}
			\item{Change directory to {\tt /tmp/jetpack-sdk-\{hash\}/packages/\{Package:name\}/}}
			\item{Call {\tt cfx xpi}.\\
				The {\tt \{Package:name\}.xpi} file will be created in current directory.} 
			\item{Send location to the front-end to be used in further actions\\
				In example calling the {\em FlightDeck Addon}\footnote{FlightDeck Addon is a Jetpack extension 
				allowing to temporary installation of the \xpi. It needs to be called with an URL of 
				the \xpi.} to download and install the \xpi.}
		\end{enumerate}
	
	\subsection{Uploading to AMO}
	
		\xpi\ needs to be created from a database object. Then use {\tt mechanize} lib to login 
		to \amo and upload the file faking it was done directly from the browser.
							
\section*{To be continued\ldots}
\end{document}

